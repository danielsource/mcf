%---------------------------------------------------------------------------------------
%	Define a caixinha para o ambiente  para os Teoremas
%---------------------------------------------------------------------------------------
\theoremstyle{plain}
\newtcbtheorem[number within=chapter]{teorema}{\strut Teorema}{
	enhanced,
	colframe=NordSnowStorm1,
	colback=white,
	colbacktitle=NordSnowStorm1,
	coltitle=NordPolarNight1,
	boxed title style={},
	attach boxed title to top left={xshift=5mm,yshift*=-\tcboxedtitleheight/2},
	before skip=10pt plus 2pt,
	after skip=10pt plus 2pt
}{th}

%---------------------------------------------------------------------------------------
%	Define a caixinha para o ambiente  para os Lemas
%---------------------------------------------------------------------------------------
\theoremstyle{plain}
\newtcbtheorem[number within=chapter]{lema}{\strut Lema}{
	enhanced,
	colframe=NordSnowStorm2,
	colback=white,
	colbacktitle=NordSnowStorm2,
	coltitle=NordPolarNight1,
	boxed title style={},
	attach boxed title to top left={xshift=5mm,yshift*=-\tcboxedtitleheight/2},
	before skip=10pt plus 2pt,
	after skip=10pt plus 2pt
}{lm}

%---------------------------------------------------------------------------------------
%	Define a caixinha para o ambiente para os Corolários
%---------------------------------------------------------------------------------------
\theoremstyle{plain}
\newtcbtheorem[number within=chapter]{corolario}{\strut Corolário}{
	enhanced,
	colframe=NordSnowStorm3,
	colback=white,
	colbacktitle=NordSnowStorm3,
	coltitle=NordPolarNight1,
	boxed title style={},
	attach boxed title to top left={xshift=5mm,yshift*=-\tcboxedtitleheight/2},
	before skip=10pt plus 2pt,
	after skip=10pt plus 2pt
}{cl}


%---------------------------------------------------------------------------------------
%	Define a caixinha para o ambiente para as Definições
%---------------------------------------------------------------------------------------
\theoremstyle{plain}
\newtcbtheorem[number within=chapter]{definicao}{\strut Definição}{
	enhanced,
	colframe=NordFrost1,
	colback=white,
	colbacktitle=NordFrost1,
	coltitle=NordPolarNight1,
	boxed title style={},
	attach boxed title to top left={xshift=5mm,yshift*=-\tcboxedtitleheight/2},
	before skip=10pt plus 2pt,
	after skip=10pt plus 2pt
}{dF}


%---------------------------------------------------------------------------------------
%	Define a caixinha para o ambiente de observação %
%---------------------------------------------------------------------------------------
\theoremstyle{plain}
\newtcbtheorem[number within=chapter]{remark}{\strut {\normalfont\fontsize{7}{7}\sffamily\selectfont\textdbend} Observação}{
	enhanced,
	colframe=NordFrost4,
	colback=white,
	colbacktitle=NordFrost4,
	coltitle=NordPolarNight1,
	boxed title style={},
	attach boxed title to top center={yshift*=-\tcboxedtitleheight/2},
	before skip=10pt plus 2pt,
	after skip=10pt plus 2pt
}{rE}

%---------------------------------------------------------------------------------------
% Ambiente exemplo sem caixa 
%---------------------------------------------------------------------------------------
\theoremstyle{definition}
\newtheorem{exem}{\color{NordAurora3}\textbf{Exemplo}}